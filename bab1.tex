%!TEX root = ./main.tex
\chapter{PENDAHULUAN}

\section{Latar Belakang}
	\lipsum
\section{Rumusan Masalah}
Berdasarkan latar belakang yang sudah dijelaskan, penulis merumuskan beberapa masalah dalam beberapa poin yang dapat dilihat di bawah ini,
\begin{enumerate}
\item Bagaimana membangun sistem otentikasi dua faktor (2FA) dengan menggabungkan faktor Biometrik berupa \textit{Voice Identification} dan \textit{One Time Password} ?
\item Bagaimana akurasi dari sistem dua faktor yang telah dibangun?
\item Bagaimana performa sistem yang telah dibuat?
\end{enumerate}

\section{Batasan Masalah}
Untuk menjaga pembahasan agar tetap fokus, maka penulis membatasi penelitian sebagai berikut:
\begin{enumerate}
	\item Implementasi sistem merupakan simulasi sistem otentikasi saja, bukan sistem ATM secara utuh.
	\item Implementasi sistem berbasis CLI (\textit{Command Line Interaction}) yang dibuat semirip mungkin dengan sistem ATM yang sudah ada.
	\item Sistem yang dibuat tidak menggunakan kartu ATM.
	\item Sistem yang dibuat hanya memiliki 20 orang nasabah, dengan komposisi 10 orang nasabah pria dan 10 orang nasabah wanita yang dipilih secara acak di sekitar lingkungan UNSIKA.
\end{enumerate}
\section{Tujuan Penelitian}
Tujuan dilakukan penelitian ini adalah sebagai berikut:
\begin{enumerate}
	\item Membangun sistem otentikasi dengan menggunakan OTP dan Voice Identification.
	\item Mengetahui akurasi dari sistem otentikasi yang dibuat.
	\item Mengetahui performa dari sistem otentikasi yang dibuat.
	\item Menguji skema otentikasi dengan menggunakan \textit{One Time Password} dan \textit{Voice Identification}
\end{enumerate}
\section{Manfaat Penelitian}
Penulis berharap dengan dilakukannya penelitian ini dapat memberikan manfaat-manfaat sebagai berikut.
\subsection{Manfaat Teoritis}
\begin{enumerate}
	\item Membuktikan akurasi sistem otentikasi dengan menggunakan multifaktor otentikasi.
	\item Memberikan pengetahuan mengenai akurasi dari penggunaan OTP dan Voice Identification dalam sistem otentikasi.
	\item Memperluas wawasan keilmuan dengan mengimplementasikan skema otentikasi yang berbeda.
\end{enumerate}
\subsection{Manfaat Praktis}
\begin{enumerate}
	\item Membantu penulis untuk mengimplementasikan keilmuannya pada masalah nyata.
	\item Memberikan rekomendasi untuk mengatasi penipuan yang dilakukan dengan menggunakan kartu ATM.
	\item Membantu Pemerintah dalam menjaga keamanan perekonomian mikro maupun makro.
\end{enumerate}

\section{Metodologi Penelitian}

Metodologi penelitian yang akan digunakan oleh penulis adalah metodologi penelitian eksperimen. 

\section{Sistematika Penulisan}
BAB I : PENDAHULUAN

Pada bab ini dijelaskan latar belakang, rumusan masalah, batasan, tujuan, manfaat, metodologi, sistematika penulisan serta jadwal penelitian.

BAB II : LANDASAN TEORI

Pada bab ini dijelaskan teori-teori dan penelitian terdahulu yang digunakan sebagai acuan dan dasar dalam penelitian.

BAB III : METODOLOGI PENELITIAN

Pada bab ini dijelaskan metode yang digunakan dalam penelitian meliputi langkah kerja, pertanyaan penelitian, objek penelitian dan alur penelitian .

\section{Jadwal Penelitian}

\begin{comment}
\bibliography{daftarpustaka}
\end{comment}